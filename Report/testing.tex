\chapter{Testing}

%==%
\section{Testing methodology and motivation}
In testing the implementation the goal is to see how performance is affected by various factors. This includes testing
for different problem configurations to see which parameters have the biggest impact, and testing runtime for the
different kernels to determine which ones among them are the most demanding.
The parameters that are tested for are \emph{number of iterations} $N_{iterations}$, \emph{grid resolution in all
dimensions}$n_x, n_y \text{and} n_z$, and \emph{number of particles} $N_{particles}$.
\subsection{Procedure}
\label{sec:testing-methodology}
Kernel timing is done using the following procedure based on the example by Mark Harris from Nvidia on \href{http://devblogs.nvidia.com/parallelforall/how-implement-performance-metrics-cuda-cc/}{devblogs.nvidia.com/...}.
\begin{lstlisting}
	cudaEvent_t beginning, end;
	cudaEventCreate(&beginning);
	cudaEventCreate(&end);
	
	cudaMemcpy(...);
	
	cudaEventRecord(beginning);
	// Code that should be measured goes in here.
	kernel<<<...>>>(...);
	cudaEventRecord(end);
	
	cudaMemcpy(...);
	
	cudaEventSynchronize(end);
	float timing = 0.0f;
	cudaEventElapsedTime(&milliseconds, beginning, end);
\end{lstlisting}
The code above works in the following way: cudaEventRecord(beginning) records the time of the next event recorded, which
would be the kernel call. The next cudaEventRecord(end) records when the next cuda event occurs, which would be the
cudaMemcpy call. cudaEventSynchronize(end) blocks CPU execution until the event has been recorded, ensuring a correct
measurement. By placing the code to be measured between cudaEventRecord(beginning) and cudaEventRecord(end) we can find
$\Delta t = end - beginning$.

Other tests are timed using QueryPerformanceCounter functions provided by Windows:
\begin{lstlisting}
	_int64 t1;
	
	QueryPerformanceCounter( (LARGE_INTEGER*)&t1 );
	//
	//code to be timed goes here
	//
	_int64  t2, ldFreq;
	QueryPerformanceCounter( (LARGE_INTEGER*)&t2 );
	
	QueryPerformanceFrequency( (LARGE_INTEGER*)&ldFreq );
	double result = ((double)(t2 - t1) / (double)ldFreq) * 1000.0;
	
	//or
	
	_int64 t1;

	StartTimer(&t1);
	//
	//code to be timed goes here
	//
	double result = StopTimer(t1);
\end{lstlisting}
Helper functions called StartTimer and StopTimer are used for simplicity, and to avoid code duplication.

%%==%
\section{System configuration}
The test system configuration is given in the figures below.
\subsection{Hardware}
The relevant hardware of the test system is given in table \ref{tab:hardware}.
\begin{table}[!htbp]
\begin{sloppypar}
	\begin{tabular}{|| p{1cm} | p{5cm} | p{5cm} ||}
	\hline
		CPU:&		\href{http://ark.intel.com/products/65523}{Core i7-3770K,\newline Intel} &
			3.50GHz $\times$ 4(8) \\ \hline
		RAM:&	\href{http://www.corsair.com/en/vengeancer-16gb-dual-channel-ddr3-memory-kit-cmz16gx3m2a1600c9g}{Corsair Vengeance 16 GB} &
			2 $\times$ 8192 MB DDR3 1600MHz, 667MHz max bandwidth\\ \hline
		MB:&		\href{http://www.asus.com/Motherboards/P8Z77V_PRO/}{P8Z77-V PRO,\newline ASUStek Computer Inc.} &
			-\\ \hline
		GPU:&		\href{http://www.gigabyte.com/products/product-page.aspx?pid=4319\#ov}{Nvidia GeForce 660ti, Gigabyte OC} &
			1344 Kepler CUDA Cores, 915Mhz, 2GB GDDR5\\ \hline
		PSU:&		\href{http://www.corsair.com/en/hx-series-hx650-power-supply-650-watt-80-plus-gold-certified-modular-psu}{Corsair HX650} &
			650W\\
	\hline
	\end{tabular}
\end{sloppypar}
%\begin{sloppypar}
%	\begin{tabular}{|| p{1cm} | p{5cm} | p{5cm} ||}
%	\hline
%	Machine:& 	MSI GE60 2PE APACHE PRO & \quad \\ \hline
%	CPU:&		Intel Core i7-4710HQ &	2.50GHz $\times$ 4(8) \\ \hline
%	RAM:&	8GB & 8192 MB DDR3 1600MHz, \\ \hline
%	GPU:&		Nvidia GeForce GTX 860M & Compute capability 5.0, 640 CUDA Cores, 1020Mhz, 2GB GDDR5 2505MHz\\
%	\hline
%	\end{tabular}
%\end{sloppypar}
\caption{Hardware used in testing.}
\label{tab:hardware}
\end{table}

\subsection{Software}
Relevant software is listed in table \ref{tab:software}.
\begin{table}[!htbp]
\begin{sloppypar}
	\begin{tabular}{|| p{3.5cm} | p{8cm} ||}
	\hline
		OS:& Windows 8.1\\
		C/C++ compiler:& MS C/C++ Optimizing Compiler v17.0.60610.1\\
		CUDA Toolkit version:& 6.5\\
		CUDA compiler:& nvcc 6.5.13\\
		GeForce driver:& 344.75\\
	\hline
	\end{tabular}
\end{sloppypar}
\caption{Software used in testing.}
\label{tab:software}
\end{table}

%==%
\section{Test parameter values}
\label{sec:testing-parameters}
\subsection{Constants}
Most of the parameters remain more or less fixed across tests. This is mainly because while they affect the numerical
accuracy of the simulation, they do not affect runtime performance. The values selected for these parameters are mostly
the same as those used by Larsgård \cite{larsgaard07}.
\begin{align*}
\intertext{\textbf{Physical constants}\newline
	\textit{Value of $\pi$ used in calculations:}}
		\pi &= 3.14159265359,\\
	\intertext{\textit{Value of permittivity of free space (electric constant):}}
		\epsilon_0 &= 8.854187817 \cdot 10^{-12 } F/m,\\
	\intertext{\textit{Value of electron charge (unit charge?):}}
		e_{charge} &= -1.60217657 \cdot 10^{-19 } C,\\
	\intertext{\textit{Value of electron mass:}}
		e_{mass} &= 9.10938291 \cdot 10^{-31 } kg,\\
\intertext{\textbf{Simulation parameters}\newline
	\textit{ Simulation grid size:}}
		L_x = L_y = L_z &= 0.2 m,\\
	\intertext{\textit{Time step between iterations:}}
		\Delta t &= 1 \cdot 10^{-6 } s,\\
	\intertext{\textit{Drag term:}}
		d_{drag} &= 0,\\
\intertext{\textbf{SOR settings}\newline
	\textit{SOR relaxation factor:}}
		\omega &= 1.78,\\
	\intertext{\textit{Error threshold for convergence (currently unused, fixed nr of iterations):}}
		Err_{threshold} &= 1 \cdot 10^{-9},\\
	\intertext{\textit{Number of SOR iterations run:}}
		N_{SOR\_iterations} &= 128,\\
\intertext{\textbf{Test specific}\newline
	\textit{Grid resolution (nr of points in each direction):}}
		n_x = n_y = n_z &= \text{default: } 64,\\
	\intertext{\textit{Simulation iterations run:}}
		N_{iterations} &= \text{default: } 2048,\\
	\intertext{\textit{Number of particles simulated:}}
		N_{particles} &= \text{default: } 256
\end{align*}

\subsection{Variables}
The testing parameters are $N_iterations$, $N_{particles}$ and $n_x,y,z$. Other than for tests involving the number of
iterations only the simulation loop is timed, the duration of one iteration of the simulation. Default values are $N_{particles} = 256$
and $n_x,y,z = 64$. The FFT based solver is chosen as default.

Certain simulation settings are derived from these terms, in particular the threadPerBlock and blockPerGrid settings for
each kernel. These depends on either the grid resolution or the particle count, as shown in section \ref{sec:implementation-setup}, and
while they are not testing parameters, they will vary as part of the testing.

%==%
\section{Description of tests}
Descriptions of all tests done are given below. These include definitions of the parameters tested and their values, as
well as a brief description of the motivation behind doing the specific test.

Performance will be measured as runtime of the code being tested. Time will be measured using the procedure given in
section \ref{sec:testing-methodology} (see also \href{http://devblogs.nvidia.com/parallelforall/how-implement-performance-metrics-cuda-cc/}{devblogs.nvidia.com/}\footnote{\url{http://devblogs.nvidia.com/parallelforall/how-implement-performance-metrics-cuda-cc/}}).

\subsection{Number of iterations}
This test just shows the increase in runtime as a function of the number of iterations.This should scale linearly with
$T(N_{iterations}) = N_{iterations}\cdot T(1)$, and a deviation from this would likely indicate something being wrong, or
at least interesting.

\testtable
	[iterations2]
	{Iterations}
	{$T_{simul}(N_{iterations}) -$ Simulation time, time spent in the simulation loop.}
	{$N_{iterations} -$ Number of iterations.}
	{$N_{iterations} \in [0, 2048]$}
	{This test serves only as a confirmation that runtime increases linearly with the number of iterations.}

\subsection{Grid resolution}
The resolution of the simulation grid ais likely one of a few key factors in determining performance. While we would
usually set $n_x = n_y = n_z$ and the effect of these on simulation accuracy should be equal in theory, the way the
arrays are stored in memory is different. For instance any $n_x \in [1,16]$ would result in the same memory footprint
due to padding (here assuming elements 8 byte wide and a 128 byte alignment requirement).

\testtable
	[grid1]
	{Grid - Isotropic scaling of $n$}
	{$T(n_x = n_y = n_z) -$ Application runtime.}
	{$n_x, n_y, n_z -$ Resolution in each dimension.}
	{$n_x = n_y = n_z \in [0,256]$}
	{Several of the kernels use one thread per grid element, and this metric is an important one. Two competing effects
	makes this an interesting one; a low resolution grid means less work to be done, potentially leaving multiprocessors on the
	card idle while reducing the number of memory accesses, but a high resolution one results in fewer particles per cell,
	thus potentially reducing write conflicts from particle-based kernels. See also test \ref{test:particles2}}

\testtable
	[grid3]
	{Grid - Odd and even valued $n$}
	{$T(n_x = n_y = n_z) -$ Application runtime.}
	{$n_x, n_y, n_z -$ Resolution in each dimension.}
	{$n_x = n_y = n_z = 2\cdot k, k \in [0,128]$
	$n_x = n_y = n_z = 2\cdot k+1, k \in [0,128]$}
	{Simulation time using the FFT solver with grid sizes even and odd separated, to easier compare performance.}
	
\testtable
	[grid2]
	{Grid - Scaling of $n$ along one dimension.}
	{$T(n_x = n_y, n_z) -$ Application runtime.}
	{$n_z -$ Resolution in z dimension.}
	{$n_x = n_y \in [0, 256]$\newline
	 $n_z \in [0, 256]$}
	{Here we test the effects on performance of scaling $n$ along a single dimension, for a different configuration of the
	other two dimensions.}

\subsection{Particle count}
First a measure of the impact the number of particles has on execution time, and then we see how different particle
counts perform for different grid resolutions.
\testtable
	[particles1]
	{Particles1 - Performance impact of particle numerosity.}
	{$T(N_{particles}) -$ Application runtime.}
	{$N_{particles} -$ Number of particles.}
	{$N_{particles} \in [0, 65536]$}
	{The number of particles affect device saturation; few particles mean less work to be done, fewer competing reads/writes,
	  while a higher number of particles mean a larger number of multiprocessors will be kept busy.}

\testtable
	[particles2]
	{Particles2 - Variation of both resolution and numerosity.}
	{$T(N_{particles}, n) -$ Application runtime.}
	{$N_{particles} -$ Number of particles.\newline
	  $n_x = n_y = n_z -$ Resolution in each dimension.}
	{$N_{particles} = \in [0, 65536]$\newline
	  $n_x = n_y = n_z \in [0, 256]$}
	{It can be interesting to see how particle number and grid resolution together affect performance, such as which
	  parameter affects performance the most. Given that the complexity of several operations is $O(n_x\cdot n_y \cdot n_z \cdot N_{particles})$
	  makes it interesting to see how performance scales when both are increased.}

\subsection{Comparison}
This pair of benchmarks time a single execution of each solver for varying grid sizes. The impact the grid size has on
their performance will be interesting to see, as well as which has better performance scaling.
\testtable
	[solvers]
	{Solvers - Comparing the runtime of the solvers.}
	{$T_{solver}(Solver, n) -$ Solver execution time}
	{The type of solver used.\newline
	  $n -$ Grid resolution.}
	{Solver $-$ cuFFT-based solver, SOR-solver.\newline
	  $n \in [0, 256]$}
	{By testing both solvers for various problem sizes we can get a measure of how well they scale, as well as which of them
	  offers better performance.}
	  
\subsection{Kernel runtimes}
These test involves measuring runtime for each kernel as functions of resolution and particle count. By timing each kernel
and comparing these results with the solver results we can see where bottlenecks are, and may learn how the different
kernels behave with variations of the simulation parameters. This is useful to identify what optimizations can be made
and which kernels have the most potential.
\testtable
	[kernel]
	{Kernels - Variation of resolution and particle numerosity.}
	{$T_{kernel}(n, N_{particles}) -$ Kernel execution time.}
	{$n_x, n_y, n_z -$ Grid resolution.\newline
	  $N_{particles} -$ Number of particles.}
	{$n \in [0, 256]$\newline
	  $N_{particles} \in [0, 65536]$}
	{Examining the performance for the different kernels for various configurations of particles and resolution, we may be
	able to identify potential bottlenecks. These results may assist in explaining global results above.}

%==%
\section{Suggestions for further testing}\label{sec:testing-further}
\subsection{Solver}
It would be useful to measure which of the two solvers offer better accuracy per performance. A test here could be to
set the number of SOR iterations so both solver use the same amount of time and then compare the numerical accuracy of
the results, but to do this we need to be able to tell which solver gives the best result, knowledge I do not have at
the present.

Assuming that the FFT as an exact solver gives the correct answer, we could run both and then find a measure of the
error by taking the average of the difference between the result for each element. By plotting this measure against grid
resolution and SOR iterations we could study how these increase the accuracy of the SOR. Combining this with a plot of
performance as a function of SOR iterations and grid resolution, we could see accuracy gain compared to performance loss,
an interesting metric.

\subsection{Single precision data type}
If the implementation is extended to include an option of using single precision data, it would be useful to test the
speedup this would gain us, as well as the inevitable loss of accuracy. The results of these tests could then help
give an idea of the trade-off involved in choosing one or the other.

For the single precision implementation we would also want to run most of the tests described above. At least for the
GeForce series of GPUs, throughput for single precision operations is significantly higher than for double precision
operations, more so than the reduction in bandwidth congestion. For this reason memory latency could become even more of
a factor, potentially reducing actual speedup, and test results might paint a different picture.

\subsection{System}Other tests that could be of interest are comparisons between running the same configuration on different systems. In
particular this goes for the GPU involved. The Tesla series of GPUs have significantly better double precision capabilities
compared to the GeForce series, relative to their single precision performance. A configuration that would fit on both
systems should perform better on the Tesla because of the number of available double precision floating point units,
even taking into account core, memory and bus clock speeds.
%==%
