\chapter{Introduction}
\pagenumbering{arabic}

%%==%
This work has examined the application of GPU acceleration of \emph{Particle-In-Cell codes}, a particle simulation scheme.
Based on previous work using Pthreads, MPI and OpenMP, this work has used Nvidia's CUDA framework to parallelize the
simulation.

The particle-in-cell method consists of modeling charged particles in an electric field, with the solving of Poisson's
equation being an important step. With applications such as simulating the behavior of plasma or modeling particle
acceleration, there is an interest in performing these simulations with high accuracy, and with a good performance so as
to enable longer simulations etc.

Because of the limited resources available for this project, the scope has been more on issues encountered when developing
the simulator, rather than optimizing it for industrial or scientific use. Some questions serving as a goal for the project are:

\begin{enumerate}
	\renewcommand{\theenumi}{\alph{enumi}}
	\item Can Particle-In-Cell simulations be implemented in CUDA? Easily? \label{list:introduction-ease}
	\item Which issues are there when mapping the problem to CUDA? Which solutions or alternatives are available? \label{list:introduction-issues}
	\item Which parts of the simulation turn out to be critical paths? Can these parts of the code be improved further? \label{list:introduction-bottleneck}
\end{enumerate}

%==%
\section{Motivation}
\subsection{Why parallelize PIC codes?}
Particle-in-cell codes lends themselves reasonably well to parallelization, and the algorithm itself is quite simple,
with few steps. With few dependencies between elements, the sequential part of computation need not be to large, allowing
us to parallelize the algorithm without getting dominated by sequential code (see \ref{sec:background-speedup}).
\subsection{Why parallelize with GPGPU?}
GPGPU is a technology that makes use of a large number of slower cores to enable massive computational throughput, and
can be very cost effective compared to scaling with CPUs. Since many steps of the PIC algorithm consist of running the
same computation on all elements in a large two or three dimensional grid, GPGPU seems to be a great fit.

%==%
\section{Related Work}
A brief introduction to previous work that serves as a base for this project.

The overarching goal for the project has been to re-implement the algorithm from my advisor Anne Cathrine Elster's Ph.d.
Parallelization Issues and Particle-In-Cell Codes (Cornell, 1994) \cite{elster94} in CUDA. This work has therefore served
as an important theoretic background during development. Elster implemented a parallel PIC simulation with Pthreads
that was aimed at the distributed memory computer KSR1. Due to hardware limitations the implementation was in 2D.
The work explains the physics behind the simulation, and goes into some detail on how to optimize for performance on a
distributed shared memory machine. Elster also demonstrated how the simulation could reproduce physical phenomena such as
plasma wave oscillation when sufficient simulation parameters were chosen. While both SOR and FFT solvers \ref{sec:background-solvers}
are described and explained, the implementation made use of the FFT solver.

Jan C. Meyer \cite{meyer04} implemented PIC codes using MPI, as part of a project to compare cluster computer applicability to
supercomputer problems. The simulation is based on Elster's previous work, but uses the SOR solver instead, and rewrote
code from scratch due to problems in converting the existing work. Issues with the implementation are discussed, and it
is suggested that the implementation can be optimized quite a bit.

This was the initial goal of the masters project of Nils M. Larsgård \cite{larsgaard07}, but focus was switched to
optimizing Elster's original code. The main topic of the work is the performance of hybrid OpenMP/MPI code, and like
Meyer's code uses MPI for communication between nodes. While the FFTW library was used to improve the FFT solver
performance, OpenMP was used to accelerate the SOR solver within nodes.

%%==%
\section{Thesis Outline}
\paragraph{Introduction}
This section, an introduction to the work, and the motivation behind it.
\paragraph{Background}
Some background on the physics behind Particle-In-Cell codes, along with some insight into the two solvers used. In
addition of the recent history in parallel computing is given, with a special focus on GPGPU and CUDA.
\paragraph{Implementation}
Details about the implementation. Some known problems and issues are mentioned, as well as what can be done to improve
upon the current implementation.
\paragraph{Testing}
Various aspects of the implementation are benchmarked. Details about the testing methodology, and suggestions for further
tests to be done in the future.
\paragraph{Results}
Performance resuslts from benchmarks.
\paragraph{Discussion}
Interpretation of results and evaluation of the implementation. Existing flaws and potential for improvement is discussed.
\paragraph{Conclusion}
A summary of findings, and ideas for future work.