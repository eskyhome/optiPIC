\chapter{Implementation}
% ==Structure== %
\section{Architecture}
Describe the structure and idea behind the program written.


% % =Simulation= %
\subsection{Simulation}
Explain how the simulation works, 


% % =Kernels= %
\subsection{Kernels}
Description of all kernels and any special handling needed.

% % determineChargesFromPotential %
\subsubsection{determineChargesFromPotential}
The kernel determining the charge density of the mesh from particle charges.
\begin{lstlisting}
\end{lstlisting}

% % electricFieldFromPotential %
\subsubsection{electricFieldFromPotential}
The kernel finding the electric field at a position given the potential.

% % updateParticles %
\subsubsection{updateParticles}
The kernel updating position and velocity of the particles.
\paragraph{electricFieldAtPoint}
Helper function that determines the electric field at any position.


% =cuFFT= %
\subsection{cuFFT}
Decribe how cuFFT setup is performed, and how the FFT is run.
% solve %
\paragraph{solve}
The kernel solving the potential in frequency space.

% =SOR= %
\subsection{SOR}
Explain how the SOR is performed.
% SOR kernel %
\paragraph{SOR kernel}
Describe the SOR kernel itself.

% =Setup= %
\subsection{Setup}
Show how setup is performed: memory allocation, constants, parameters.


% ==Performance== %
\section{Performance}
Uncertain what I intended to put here. Perhaps description of steps taken to
improve performance, such as shared memory, aligned memory allocations 
(pitched pointers), perhaps data types, and any other optimizations...
% =Data= %
\subsection{Data}
% =Solvers= %
\subsection{Solvers}
% FFT %
\subsubsection{FFT}
% SOR %
\subsubsection{SOR}

% =Particle Tracing= %
\subsection{Particle tracing}
Brief description of how tracing is performed, and how it will affect
performance? Perhaps also any ideas for how this can be improved in the future.
