\chapter{Background}

% ==Particle-In-Cell Codes== %
\section{Particle-In-Cell Codes}
A technique for solving certian PDEs. Usually involves simulating particles or
fluid in a grid. Also for plasma simulations.
\paragraph{Applications}
Used for studying plasma, electrons, ions. Other applications in solid/fluid
mechanics.

% =Physics= %
\subsection{Physics}
This section should describe the physical basis for the simulation.

% Subject of simulation %
\subsubsection{Subject of simulation}
What is being modeled?
A set of particles with position velocity mass and charge. Attempting to
determine behaviour over time. Need to find force on particles. Need to
determine electric field from potential. Find potential from distribution of
charge from particles. Repeat. Simple.

Discrete in space and time. Details below.

% Equations %
\subsubsection{Discrete Model Equations}
Mention that these are the equations used in Elsters phd...
\begin{equation} \nabla^2\Phi = -\frac{\rho}{\epsilon _0} \end{equation}
\begin{equation} E = -\nabla\Phi \end{equation}
\begin{equation} \frac{dx}{dt} = v \end{equation}
\begin{equation} \frac{dv}{dt} =\frac{qE}{m} \end{equation}

\paragraph{Poissons equation}
$$ \nabla^2\Phi = -\frac{\rho}{\epsilon _0}$$
$$\frac{\partial ^2\Phi}{\partial ^2 x} + \frac{\partial^2\Phi}{\partial ^2 y} +
\frac{\partial ^2\Phi}{\partial ^2 z}= -\frac{\rho}{\epsilon _0}$$
Poissons equation: a PDE that relates the charge to the Laplacian of the potential.
\subparagraph{Describe discretization implications}

\paragraph{Electric field from Potential}
$$ E = -\nabla\Phi $$
Relates the electric field strength to the potential. The electric field
strength at a location is proportional to the gradient of the potential,
and is sampled at each grid vertex by taking the difference in each direction.

\paragraph{Electric force from the electric field}
$$ F_\text{electric} = qE $$
Electrical force is defined as the field strength $E$ times the charge $q$.

\paragraph{Acceleration, velocity - Electric force}
$$ \frac{dv}{dt} = \frac{qE}{m} $$
$$ \frac{dx}{dt} = v $$
Knowing that the acceleration of a particle $a = \frac{F}{m} = \frac{qE}{m}$,
we can estimate the velocity and position as $v = v_0 + a\cdot\Delta t$,
$p = p_0 + v\cdot\Delta t$.
\subparagraph{Describe discretization (time, space)}

% Procedure %
\subsubsection{Procedure}
\begin{enumerate}
	\item Find charge density.
	\item Solve for potential.
	\item Determine electric field from potential.
	\item Update particles.
\end{enumerate}

% ==Solvers== %
\section{Solvers}
% Boundary conditions %
\subsection{Boundary conditions}
Outline which different boundary conditions can appear, and for which ones the
different solvers are suited.

% =FFT= ++%
\subsection{FFT}
Short description of the concept behind the FFT.
\subsubsection{Fourier Transform}
Brief explanation of the math behind  the fourier transform.
\subsubsection{DFT}
How the continous transform translates to discrete applications.
\subsubsection{Cooley-Tukey FFT algorithm}
On the C-T FFT.
\subsubsection{Performance}
NlogN complexity, etc.
\subsubsection{Properties}
Aliasing etc.
\subsubsection{Applications}
Some common applications of the FFT, and reasons it is used.

% SOR %
\subsection{SOR}
\subsubsection{System of Linear Equations}
Properties of a system of linear equations. Also how this problem is represented
as one.
\subsubsection{Gauss-Seidel}
Efficient serial solver. How it translates to paralell.
\subsubsection{Over-relaxation}
How this works, and how it improves performance.
\subsubsection{Performance}
Some statistics to compare SOR to other solvers, reference point for comparison.
\subsubsection{Properties}
Tendency to "float" under certain boundary conditions, inconvergence etc.
\subsubsection{Applications}
Examples of successful application of SOR as a solver, and reasons it was used.


%% MOVE TO MOTIVATION? %%

% ==Parallel Computing== %
\section{Parallel Computing}
This section explains the motivations behind parallel programing and GPGPU in
general. It should show general characteristics of the CUDA architecture and
programming model, and show the contrast to CPU programming, mentioning both the
benefits and trade-offs, and reasons it is different.

% =Parallel computing basics %
\subsection{Parallel computing basics}
Briefly explain the essence of TDT4200, as much as is necessary to understand
the rest of the report.
\paragraph{Moore's law}
Outline regular increase in computing power as transistor density decreased.
\paragraph{Power wall, etc}
Explain why new measures had to be taken to increase performance further,
leading to increasing focus on multicore.
\paragraph{Amdahls law, Gustafsons law}
Explain what Amdahl and gustafsons laws mean for paralellization of serial
programs. How to bypass the problems by inccreasing problem size.
% ...and anything else worth mentioning. %
%%%%


% =GPGPU= %
\subsection{GPGPU}
Brief history of GPGPU, mention relationship to gustafsons law (increase
problem size at cost of processing speed). Architechture and language agnostic,
what are the benefits and costs of this approach.

% History %
\subsubsection{History}
How did GPGPU come about, and how has it developed since. Status today.

% Purpose %
\subsubsection{Purpose}
Mathematical/scientific reasons that GPGPU performs well, and why it is used.

% =CUDA= %
\subsection{CUDA}
Specifics on the cuda preogramming language/architecture. It's origin and
development since.

% Programming model %
\subsubsection{Programming model}
How to write a CUDA kernel. Why one writes a kernel. How a kernel differs from
any other funciton call.

% Architecture %
\subsubsection{Architecture}
The CUDA architecture and how it must be taken into consideration when writing
CUDA programs.
\paragraph{General}
Genral layout, function and performance of the CUDA architecture.
\paragraph{Maxwell}
Specifics for the Kepler and Maxwell generations.

% Memory %
\subsubsection{Memory}
Memory management. Different kinds of memory, allocating memory, atomic
operations etc...

% cuFFT %
\subsubsection{cuFFT}
Details on the cuFFT framework. How it is used, performance compared to FFTW...


%Previous work%
\section{Previous work}